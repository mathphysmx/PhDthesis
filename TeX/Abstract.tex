\chapter*{Abstract}
\addcontentsline{toc}{chapter}{Abstract}
\markboth{Abstract}{Abstract} 	
The main motivation of this work was to develop a methodological approach for the characterization, modeling and simulation of Discrete Fractures Networks (DFN) in 2D that considers the dependence structure between the properties of the fracture properties. For modeling of dependence, copula theory was used with the non-parametric approach of Bernstein's copulas. This approach allowed explicit consideration of the orientation-length and orientation-length-aperture relationships in the modeling, since the orientation, being a directional data, imposes new restrictions to the conventional methodology. Such relationships can have relevant consequences on the percolation properties and consequently on the flow and transport properties of the system being studied.

The approach of discrete fracture networks consists in applying a Boolean stochastic simulation method, also known as an object-based simulation method, where fractures are represented as simplified geometric objects (2D line segments and 3D polygons). In this way, fractures are considered explicitly and can be included in the flow and transport models when these fractures are relevant.

Within the geological-petrophysical reservoir modeling, and from a methodological point of view, it is described in detail the modeling of each of the elements of a discrete fracture network with both the standard methodology and the proposed methodology. Emphasis was also placed on showing flowcharts within each stage of modeling as well as showing their relationship with numerical and computational models.

This methodology was applied in two fracture networks generated synthetically but respecting geological arguments taken from the literature. In the first case study, only the direction-length relationship is modeled and in the other, the triad direction-length-aperture is modeled. In both examples, it was shown that the common approach where the dependency is not considered generates DFNs that are very far from the actual data when there is dependence. In particular, it is shown that with the conventional approach the univariate behavior of each fracture property can be reproduced, but the bi- or trivariate behavior is totally different from that of the copula theory, which actually reproduces the dependence structure of the data.
