% random Variable simulation flowchart
%https://www.sharelatex.com/blog/2013/08/29/tikz-series-pt3.html
%http://tex.stackexchange.com/questions/44505/export-tikz-figures-to-pdf
\documentclass[border=1pt]{standalone}
\usepackage{tikz}
\usetikzlibrary{shapes.geometric, arrows}

\begin{document}
% https://www.sharelatex.com/blog/2013/08/29/tikz-series-pt3.html
\tikzstyle{startstop} = [rectangle, rounded corners, minimum width=3cm, minimum height=1cm,text centered, draw=black, fill=red!30]
\tikzstyle{io} = [trapezium, trapezium left angle=70, trapezium right angle=110, minimum width=3cm, minimum height=1cm, text centered, draw=black, fill=blue!30]
\tikzstyle{process} =  [rectangle, minimum width=3cm, minimum height=1cm, text centered, draw=black, fill=lightgray]
\tikzstyle{decision} = [diamond,   minimum width=3cm, minimum height=1cm, text centered, draw=black, fill=green!30]
\tikzstyle{arrow} = [thick,->,>=stealth]

\begin{tikzpicture}[node distance=2cm]
%% c2D stands for 'c'ounts of inside each bin of a 2D histogram.
%% computationally, c2D is a 2D array, i. e., a matrix
	\node (in1) [startstop, fill = green] {c2 = Generar histograma 2D de las pseudo-observaviones};
	\node (pro1) [process, below of=in1] {Agregar peridicidad $p_{ij}$ a c2};
	\node (out) [startstop, below of=pro1] {Recuperar la c\'opula empírica $\hat{C_n}$ a partir de c2};
\draw [arrow] (in1) -- (pro1);
\draw [arrow] (pro1) -- (out);
\end{tikzpicture}

\end{document}

